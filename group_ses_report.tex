The global probabilities partially align with personal preferences. While several globally favored attributes such as \textbf{bright timbre} and \textbf{non-party contexts} are also reflected in personal 5-star ratings, certain globally dominant artists and popularity effects do not consistently appear in individual sessions, as well as the release year of preferred tracks in out personal sessions doesn't add to a significant count, and is thus omitted.. This suggests that personal taste captures only a subset of global trends, with individual deviations playing a significant role.

Personal sessions exhibit stronger and more polarized conditional probabilities compared to global and group-level models. While global tendencies emphasize popularity and broadly favored audio features, personal sessions reveal sharper preferences, particularly in timbre and mood dimensions. Group-level averages smooth out these individual effects, retaining only the most consistent attributes, thereby highlighting the inherently personalized nature of music preference.
